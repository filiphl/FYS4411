\documentclass[english, a4paper]{article}

\usepackage[T1]{fontenc}    % Riktig fontencoding
\usepackage[utf8]{inputenc} % Riktig tegnsett
\usepackage{babel}          % Ordelingsregler, osv
\usepackage{graphicx}       % Inkludere bilder
\usepackage{booktabs}       % Ordentlige tabeller
\usepackage{url}            % Skrive url-er
\usepackage{textcomp}       % Den greske bokstaven micro i text-mode
\usepackage{units}          % Skrive enheter riktig
\usepackage{float}          % Figurer dukker opp der du ber om
\usepackage{lipsum}         % Blindtekst
\usepackage{subcaption} 
\usepackage{color}
\usepackage{amsmath}  
\usepackage{minted}
\usepackage{braket} 
\usepackage{multicol}
\usepackage{listings}    %Add source code
\usepackage{amsfonts}
\usepackage{setspace}
\usepackage[cm]{fullpage}		% Smalere marger.
\usepackage{verbatim} % kommentarfelt.
\setlength{\columnseprule}{1pt}	%(width of separationline)
\setlength{\columnsep}{1.0cm}	%(space from separation line)
\newcommand\lr[1]{\left(#1\right)} 
\newcommand\bk[1]{\langle#1\rangle} 
\newcommand\uu[1]{\underline{\underline{#1}}} % Understreker dobbelt.
\definecolor{qc}{rgb}{0,0.4,0}
\definecolor{LightBlue}{rgb}{0.8, 0.8, 0.9}

% JF i margen
\makeatletter
\renewcommand{\subsubsection}{\@startsection{subsubsection}{3}{0pt}%
{-\baselineskip}{0.5\baselineskip}{\bf\large}}
\makeatother
\newcommand{\jf}[1]{\subsubsection*{JF #1}\vspace*{-2\baselineskip}}

\newcommand{\bm}[1]{\mathbf{#1}}

% Skru av seksjonsnummerering (-1)
\setcounter{secnumdepth}{3}

\begin{document}
\renewcommand{\figurename}{Figure}
% Forside
\begin{titlepage}
\begin{center}

\textsc{\Large FYS4411 - Computational quantum mechanics }\\[0.5cm]
\textsc{\Large Spring 2016}\\[1.5cm]
\rule{\linewidth}{0.5mm} \\[0.4cm]
{ \huge \bfseries  Project 1;\\ Variational Monte Carlo Studies of Bosonic systems}\\[0.10cm]
\rule{\linewidth}{0.5mm} \\[1.5cm]
\textsc{\Large temporary report}\\[1.5cm]


% Av hvem?
\begin{minipage}{\textwidth}
\begin{minipage}{0.49\textwidth}
    \begin{center} \large
        Sean Bruce Sangolt Miller\\
        {\footnotesize s.b.s.miller@fys.uio.no}
    \end{center}
\end{minipage}
\quad
\begin{minipage}{0.49\textwidth}
    \begin{center} \large
        Filip Henrik Lasren\\
        {\footnotesize filiphenriklarsen@gmail.com}
    \end{center}
\end{minipage}
\end{minipage}
\vfill

% Dato nederst
\large{Date: \today}

\end{center}
\end{titlepage}
%%%%%%%%%%%%%%%%%%%%%%%%%%%%%%%%%%%

\begin{abstract}
 Some text that is abstact
\end{abstract}


%%%%%%%%%%%%%%%%%%%%%%%%%%%%%%%%%%%
\pagenumbering{gobble}% Remove page numbers (and reset to 1)
\tableofcontents
\newpage
\pagenumbering{arabic}% Arabic page numbers (and reset to 1)
%\begin{multicols*}{2}


\section{Introduction}


\section{Theory and methods}
\subsection{Preliminary derivations}

\subsubsection{Simplified problem}
The local energy is defined as:

\begin{equation}
    E_L({\bf R})=\frac{1}{\Psi_T({\bf R})}H\Psi_T({\bf R}),
    \label{eq:locale}
\end{equation}

As a first approximation, it is assumed there is no interaction term in the Hamiltonian, which means the hard sphere bosons have no physical size (the hard-core diameter is zero). It is also assumed that no magnetic field is applied to the bosonic gas, leaving a perfectly spherically symmetrical harmonic trap. Inserting this new Hamiltonian into the local energy gives:

\begin{equation}
  E_L({\bf R})=\frac{1}{\Psi_T({\bf R})}\sum_i^N \left(
  \frac{-\hbar^2}{2m}
  \nabla_{i}^2 +
  \frac{1}{2}m\omega_{ho}^2r_i^2\right)\Psi_T({\bf R})
  \label{eq:localeInsertH}
\end{equation}

The potential term is trivial since this is a scalar, i.e. the denominator will cancel the wavefunction. A more challenging problem is to find an expression for $\nabla^2_i\Psi_T({\bf R})$. The trial wavefunction shown in equation (...), with the aforementioned approximations, is:

\begin{equation}
 \Psi_T({\bf R}) = \prod_i e^{-\alpha r_i^2}	\label{wavefunction_a}
\end{equation}

where $\alpha$ is the variational parameter for VCM. The first derivative is:

\begin{align}
 \nabla_j\prod_i e^{-\alpha r_i^2} 
 &= -2\alpha \bm{r}_j e^{-\alpha r_j^2} \prod_{i \neq j} e^{-\alpha r_i^2}\\
 &= -2\alpha \bm{r}_j  \prod_i e^{-\alpha r_i^2}.
 \label{E_L_first_derivative}
\end{align}

The second derivative then follows:

\begin{align}
 \nabla_j^2 \prod_i e^{-\alpha r_i^2} 
 &= \nabla_j \lr{-2\alpha \bm{r}_j  \prod_i e^{-\alpha r_i^2}}\\
 &= \lr{4\alpha^2 r_j^2 - 2d\alpha}  \prod_i e^{-\alpha r_i^2}.
\end{align}

where $d$ is the number of dimensions. Inserting this into back into the local energy (equation \eqref{eq:localeInsertH}), the final expression can be derived:

\begin{align*}
 E_L({\bf R}) &= \frac{1}{\Psi_T({\bf R})}\sum_i^N \lr{
	 \frac{-\hbar^2}{2m}
	 \nabla_{i}^2 +
	\frac{1}{2}m\omega_{ho}^2r^2}\Psi_T({\bf R})\\
    &= \sum_{i=1}^N\left[\frac{-\hbar^2}{2m}\lr{4\alpha^2 r_i^2 - 2d\alpha} + \frac{1}{2}m\omega_{ho}^2r_i^2\right]
\end{align*}



The drift force (quantum force), still with the approximations above, is defined by:

\begin{equation}
	F = \frac{2\nabla\Psi_T}{\Psi_T}
\end{equation}

The gradient here is defined as

\begin{equation*}
	\nabla \equiv \lr{\nabla_1,\nabla_2,\ldots,\nabla_N}
\end{equation*}

i.e. a vector of dimension $Nd$. The gradient with respect to a single particle's position is already given in equation \ref{E_L_first_derivative}, so it's not too hard to realise:

\begin{align*}
	F &= \frac{-4\alpha}{\Psi_T}\lr{\bm{r}_1,\bm{r}_2,\ldots,\bm{r}_N}\Psi_T\\
	&= -4\alpha\lr{\bm{r}_1,\bm{r}_2,\ldots,\bm{r}_N}
\end{align*}


\subsubsection{Full problem}

The full local energy\footnote{The "full local energy" means not making any assumptions on the particle interactions or the potential.} is a bit more tedious to derive. The first step is to rewrite the trial wavefunction to the following form:

\begin{equation}
	\Psi_T(\bm{R}) = \prod_{i} \phi(\bm{r}_i)e^{\sum_{i'<j'}u(r_{i'j'})}
\end{equation}

where, in order for this to fit with the previous wavefunction, $u(r_{ij}) \equiv \ln[f(r_{ij})]$ and $\phi(\bm{r}_i) \equiv g(\alpha,\beta,\bm{r}_i)$. The gradient with respect to the $k$-th coordinate set is:

\begin{align}
	\nabla_k\Psi_T &= \nabla_k\prod_i \phi(\bm{r}_i)e^{\sum_{i'<j'}u(r_{i'j'})}\\
	&= \nabla_k\phi_k\left[\prod_{i\neq k} \phi(\bm{r}_i)e^{\sum_{i'<j'}u(r_{i'j'})}\right] + \left[\prod_i \phi(\bm{r}_i)e^{\sum_{i'<j'}u(r_{i'j'})}\nabla_k\left(\sum_{i''<j''}u_{i''j''} \right)\right]\\
	&= \nabla_k\phi_k\left[\prod_{i\neq k} \phi(\bm{r}_i)e^{\sum_{i'<j'}u(r_{i'j'})}\right] + \left[\prod_i \phi(\bm{r}_i)e^{\sum_{i'<j'}u(r_{i'j'})}\left(\sum_{i''<j''}\nabla_ku_{i''j''} \right)\right]
\end{align}

The function $u_{ij}$ is symmetric under permutation $i \leftrightarrow j$, as one can see from the definitions of $u_{ij}$ and $f(r_{ij})$. Therefore, the last sum above can have a different indexing: All terms without an index $k$, will give zero when taking the derivative $\nabla_k$, so only $i=k$ or $j=k$ remains (remember $i \neq j$). Due to the symmetry of $u_{ij}$, one can simply always say $i=k$ and let $j$ be the summation index:

\begin{equation}
	\nabla_k\Psi_T = \nabla_k\phi_k\left[\prod_{i\neq k} \phi(\bm{r}_i)e^{\sum_{i'<j'}u(r_{i'j'})}\right] + \left[\prod_i \phi(\bm{r}_i)e^{\sum_{i'<j'}u(r_{i'j'})}\left(\sum_{j''\neq k}\nabla_ku_{kj''} \right)\right]
	\label{eq:grad}
\end{equation}

The second derivative now becomes (where $\nabla_k$ only acts on the first parenthesis to its right):

\begin{align*}
	\nabla_k^2\Psi_T &= (\nabla_k^2\phi_k)\left[\prod_{i\neq k} \phi(\bm{r}_i)e^{\sum_{i'<j'}u(r_{i'j'})}\right] + (\nabla_k\phi_k) \left[\prod_{i\neq k}\phi(\bm{r}_i)\nabla_ke^{\sum_{i'<j'}u(r_{i'j'})}\right]\\
	&\:\:\:\:\:\: + \left[\nabla_k\left(\prod_i \left(\phi(\bm{r}_i)\right)e^{\sum_{i'<j'}u(r_{i'j'})}\right)\left(\sum_{j''\neq k}\nabla_ku_{kj''} \right)\right]\\
	&\:\:\:\:\:\: + \left[\prod_i \phi(\bm{r}_i)e^{\sum_{i'<j'}u(r_{i'j'})}\left(\sum_{j''\neq k}\nabla_k^2u_{kj''} \right)\right]
\end{align*}

While a bit of a nuisance to read, the expression above is simply the product rule for $\nabla_k(\nabla_k\Psi_T)$. Written in terms of $\Psi_T$, the above can be a bit simplified\footnote{For the more mathematically concerned nitpicker, the product symbol only runs over the functions $\phi(\bm{r}_i))$, not the following exponential. This is easy to realise by recalling how $\Psi_T$ was defined, and is important to know when inserting $\Psi_T$ as done now.}:

\begin{align*}
	\nabla_k^2\Psi_T &= (\nabla_k^2\phi_k)\frac{\Psi_T}{\phi(\bm{r}_k)} + (\nabla_k\phi_k) \left[\prod_{i\neq k}\phi(\bm{r}_i)\nabla_ke^{\sum_{i'<j'}u(r_{i'j'})}\right]\\
	&\:\:\:\:\:\: + \left[\left(\nabla_k\Psi_T\right)\left(\sum_{j''\neq k}\nabla_ku_{kj''} \right)\right]\\
	&\:\:\:\:\:\: + \left[\Psi_T\left(\sum_{j''\neq k}\nabla_k^2u_{kj''} \right)\right]
\end{align*}

The second term above is equal to the second term in equation \ref{eq:grad}, divided by $\phi(\bm{r}_k)$. Furthermore, the gradient $\nabla_k\Psi_T$ is already calculated above, and in terms of $\Psi_T$ is:

\begin{equation}
	\nabla_k\Psi_T = \frac{\Psi_T}{\phi(\bm{r}_k)}\nabla_k\phi_k + \Psi_T\left(\sum_{j''\neq k}\nabla_ku_{kj''} \right)
\end{equation}

Inserting all this back into the second derivative yields:

\begin{align*}
	\nabla_k^2\Psi_T &= (\nabla_k^2\phi_k)\frac{\Psi_T}{\phi(\bm{r}_k)} + (\nabla_k\phi_k) \left[\frac{\Psi_T}{\phi(\bm{r}_k)}\left(\sum_{j''\neq k}\nabla_ku_{kj''} \right)\right]\\
	&\:\:\:\:\:\: + \left[\frac{\Psi_T}{\phi(\bm{r}_k)}\nabla_k\phi_k + \Psi_T\left(\sum_{j''\neq k}\nabla_ku_{kj''} \right)\right]\left(\sum_{j''\neq k}\nabla_ku_{kj''} \right)\\
	&\:\:\:\:\:\: + \left[\Psi_T\left(\sum_{j''\neq k}\nabla_k^2u_{kj''} \right)\right]
\end{align*}

Giving:

\begin{align*}
	\frac{1}{\Psi_T}\nabla_k^2\Psi_T &= (\nabla_k^2\phi_k)\frac{1}{\phi(\bm{r}_k)} + (\nabla_k\phi_k) \left[\frac{1}{\phi(\bm{r}_k)}\left(\sum_{j''\neq k}\nabla_ku_{kj''} \right)\right]\\
	&\:\:\:\:\:\: + \left[\frac{1}{\phi(\bm{r}_k)}\nabla_k\phi_k + \left(\sum_{j''\neq k}\nabla_ku_{kj''} \right)\right]\left(\sum_{j''\neq k}\nabla_ku_{kj''} \right)\\
	&\:\:\:\:\:\: + \sum_{j''\neq k}\nabla_k^2u_{kj''}
\end{align*}

Which can be rewritten to:

\begin{align}
	\frac{1}{\Psi_T}\nabla_k^2\Psi_T &= \frac{\nabla_k^2\phi_k}{\phi(\bm{r}_k)} + \frac{2\nabla_k\phi_k}{\phi(\bm{r}_k)}\left(\sum_{i\neq k}\nabla_ku_{ki} \right) + \left(\sum_{j\neq k}\nabla_ku_{kj} \right)^2 + \sum_{l\neq k}\nabla_k^2u_{kl}
\end{align}

The gradients $\nabla_k u_{ki}$ can be rewritten using partial differentiation (where $r_{k,i}$ is coordinate $i$ of $\bm{r}_k$):

\begin{align*}
	\nabla_k u_{ki} &= \left(\frac{\partial}{\partial r_{k,1}}, \frac{\partial}{\partial r_{k,2}}, \ldots\right)u_{ki}\\
	&= \frac{\partial u_{ki}}{\partial r_{ki}}\left(\frac{\partial r_{ki}}{\partial r_{k,1}}, \frac{\partial r_{ki}}{\partial r_{k,2}}, \ldots\right)\\
	&= \frac{\partial u_{ki}}{\partial r_{ki}}\left( \frac{\partial}{\partial r_{k,1}}\left(\sqrt{\left[(r_{k,1}-r_{i,1})\hat{e}_1 + \ldots\right]^2}\right), \ldots \right)\\
	&= \frac{\partial u_{ki}}{\partial r_{ki}}\left(\frac{r_{k,1} - r_{i,1}}{r_{ki}}, \ldots \right)\\
	&= \frac{\partial u_{ki}}{\partial r_{ki}}\frac{\bm{r}_k - \bm{r}_i}{r_{ki}}\\
	&= \frac{\bm{r}_k - \bm{r}_i}{r_{ki}} u_{ki}' \:\:,\:\:  u_{ki}'\equiv \frac{\partial u_{ki}}{\partial r_{ki}}
\end{align*}

While the second derivative of $u_{ki}$ is:

\begin{align*}
	\nabla_k^2 u_{ki} &= \nabla_k(\nabla_ku_{ki})\\
	&= u_{ki}'\nabla_k\left(\frac{\bm{r}_k - \bm{r}_i}{r_{ki}}\right) + \frac{\bm{r}_k - \bm{r}_i}{r_{ki}}\nabla_k u_{ki}'\\
	&= u_{ki}' \left(\frac{d}{r_{ki}} - \frac{r_{k,1}-r_{i,1}}{r_{ki}^3} - \frac{r_{k,2}-r_{i,2}}{r_{ki}^3} - \ldots\right) + \frac{\bm{r}_k - \bm{r}_i}{r_{ki}}\cdot\frac{\bm{r}_k - \bm{r}_i}{r_{ki}} u_{ki}''\\
	&= \frac{u_{ki}'}{r_{ki}} \left(d - \frac{r_{k,1}-r_{i,1}}{r_{ki}^2} - \frac{r_{k,2}-r_{i,2}}{r_{ki}^2} - \ldots\right) + u_{ki}\\
	&= \frac{u_{ki}'}{r_{ki}} \left(d - 1\right) + u_{ki}
\end{align*}

where $d$, as earlier, is the number of dimensions present, which in our world is usually $d=3$. Finally, this gives:

\begin{align}
\frac{1}{\Psi_T}\nabla_k^2\Psi_T &= \frac{\nabla_k^2\phi_k}{\phi(\bm{r}_k)} + \frac{2\nabla_k\phi_k}{\phi(\bm{r}_k)}\left(\sum_{i\neq k} \frac{\bm{r}_k - \bm{r}_i}{r_{ki}} u_{ki}' \right) + \sum_{i,j\neq k}\frac{(\bm{r}_k - \bm{r}_i)(\bm{r}_k - \bm{r}_j)}{r_{ki}r_{kj}}u_{ki}'u_{kj}' + \sum_{l\neq k} u_{ki} + \frac{u_{ki}'}{r_{ki}} \left(d - 1\right)
\label{eq:full_second_derivative}
\end{align}

As the exact forms of $\phi_k$ and $u_{ki}$ are known, this can be written to a more recognisable, and calculable, expression. However, this expression will be quite long, so only the necessary variables will be derived. Inserting them into equation \ref{eq:full_second_derivative} is trivial. The $u_{ki}$ derivatives are:

\begin{equation}
	\frac{\partial u_{ki}}{\partial r_{ki}} = \frac{\partial}{\partial r_{ki}}\left(\ln\left[1-\frac{a}{r_{ki}}\right]\right) = \frac{a}{r_{ij}^2 - ar_{ij}}
\end{equation}
\begin{equation}
	\frac{\partial^2 u_{ki}}{\partial r_{ki}^2} = \frac{\partial}{\partial r_{ki}} u_{ki}' = -a\frac{2r_{ki} - a}{r_{ki}^2 - ar_{ki}}
\end{equation}

and the $\phi_k$ derivatives are:

\begin{equation}
	\nabla_k\phi_k = \nabla_k e^{-\alpha(x_k^2 + y_k^2 + \beta z_k^2)} = -2\alpha(x_k,y_k,\beta z_k)\phi_k
\end{equation}

\begin{equation}
\nabla_k^2\phi_k = \nabla_k (\nabla_k\phi_k) = \left[-2\alpha(2+\beta) + 4\alpha^2(x_k^2 + y_k^2 + \beta^2 z_k^2)\right]\phi_k
\end{equation}




\subsection{The method of steepest descent}
In order to find the value for $\alpha$ that minimizes $\langle E_L\rangle$, the method of steepest descent (SD) is applied. The SD method, in algorithm form, is:

\begin{equation}
	x_{n+1} = x_n - \gamma_n\nabla f
\end{equation}

where $x$ is the variable with which one wishes to find the minimum of $f$. In application to the current problem, $f = \langle E_L\rangle$. However, since $\langle E_L\rangle$ is an expensive quantity to find numerically, and its derivative ($\bar{E}_\alpha \equiv \frac{d\langle E_L\rangle}{d\alpha}$) even more so, an analytical expression is desirable. This can be found as follows:

\begin{align}
	\begin{split}
		\bar{E}_\alpha &= \frac{d}{d\alpha}\int dx P(x) E_L\\
		&= \frac{d}{d\alpha}\int dx \frac{|\psi|^2}{\int dx'|\psi|^2}\frac{1}{\psi}H\psi\\
		&= \frac{d}{d\alpha}\int dx \frac{\psi^*H\psi}{\int dx'|\psi|^2}
	\end{split}
\end{align}

Since the Hamiltonian is hermitian, one has $\int dx\psi^* H \psi = \int dx H\psi^*\psi$, giving:

\begin{align}
	\begin{split}
		&= \frac{d}{d\alpha}\int dx \frac{H\psi^*\psi}{\int dx'|\psi|^2}\\
		&= \left[ \int dx\frac{H\left(\psi^*\left(\frac{d\psi}{d\alpha}\right) + \left(\frac{d\psi^*}{d\alpha}\right)\psi\right)}{\int dx'|\psi|^2} \right] - \left[ \int dx \frac{H\psi^*\psi}{\left(\int dx'|\psi|^2\right)^2}\int dx'\left( \psi^*\left(\frac{d\psi}{d\alpha}\right) + \left(\frac{d\psi^*}{d\alpha}\psi\right) \right) \right]
	\end{split}
\end{align}

Again one may use the hermiticity of the Hamiltonian to get $\int dx H \psi^*\left(\frac{d\psi}{d\alpha}\right) = \int dx H \left(\frac{d\psi^*}{d\alpha}\right)\psi$. So:

\begin{align}
	\begin{split}
		&= 2\left[ \int dx\frac{H\psi^*\frac{d\psi}{d\alpha}}{\int dx'|\psi|^2} \right] - 2\left[ \int dx \frac{H\psi^*\psi}{\left(\int dx'|\psi|^2\right)^2}\int dx' \psi^*\frac{d\psi}{d\alpha} \right]\\
		&= 2\left[ \int dx\frac{H\psi^*\frac{d\psi}{d\alpha}}{\int dx'|\psi|^2} - \int dx \frac{H\psi^*\psi}{\int dx'|\psi|^2}\int dx' \frac{1}{\int dx'|\psi|^2}\psi^*\frac{d\psi}{d\alpha} \right]\\
		&= 2\left[ \int dx\frac{\psi^*\left(\frac{E_L}{\psi}\frac{d\psi}{d\alpha}\right) \psi}{\int dx'|\psi|^2} - \int dx \frac{\psi^* E_L\psi}{\int dx'|\psi|^2}\int dx' \frac{\psi^*\left(\frac{1}{\psi}\frac{d\psi}{d\alpha}\right)\psi}{\int dx'|\psi|^2} \right]\\
		&= 2\left( \langle\frac{\bar{\psi}_\alpha}{\psi}E_L\rangle -  \langle\frac{\bar{\psi}_\alpha}{\psi}\rangle\langle E_L\rangle \right)
	\end{split}
\end{align}

where $\bar{\psi}_\alpha \equiv \frac{d\psi}{d\alpha}$. This is a much better expression to use since one only need one Monte Carlo cycle to find $\bar{E}_\alpha$. Therefore, one Monte Carlo cycle will give one value for $\bar{E}_L$.\\
Since $\alpha$ will only have to be determined once, it is permissible to do so with greater accuracy. This means ... (explain about higher accuracy and divisions by 2)










\section{Results}


\subsection{Benchmarks}
As in most computational work, we need to make certain our program produce reliable values. 
A very useful way to do this is to compare our numerical results with a known analytical solution to the problem. 
Since we do know the analytical solution to the spherical harmonic oscillator, these may be used as benchmarks that we hope to reproduce.
We have already derived an analytical expression in \eqref{SimplifiedLocalEnergy}. 

\subsubsection{Standard Metropolis sampling}
Tests were done with several choices of number of particles, and number of dimensions.
The outputs were the expectation value of the energy, the standard deviation of this value and the time elapsed.

At first we chose to calculate the laplacian in the kinetic term numerically. 
The results are shown in table(blah).
As can be seen the analytical results are well reproduce. 
The standard deviation is not zero, but this is due to numerical error in the calculation of the kinetic energy.
This error increases with the number of particles and number of dimensions. 
%We know this because the potetial energy is trivial and equal the analytical expression by definition.

Once convinced the local energies were reproduced correctly, we implemented an analytical calculation of the laplacian as well.
A user may choose to use this option by setting 
\\*

\begin{minted}[bgcolor=LightBlue, fontsize=\footnotesize]{c++}
system->setAnalyticalLaplacian (true);
\end{minted}
\\*

\noindent in \texttt{main.cpp}. The tests using analytical calculation of the laplacian resulted as shown in table(blah2).

\subsubsection{Importance sampling}
Replacing the standard metropolis sampling with importance sampling, we allow a so called quantum force to affect the probability of accepting a move.
This force works as a drift toward the most probable state. 
This should effectively, as the name suggests, allow us to sample more important areas and thereby not discard as many moves.
In the same fashion as the toggle for use of analytical calculation of the laplacian, one can choose to use importance sampling by setting 
\\*

\begin{minted}[bgcolor=LightBlue, fontsize=\footnotesize]{c++}
system->setImportanceSampling (true);
\end{minted}
\\*

\noindent in \texttt{main.cpp}. Results using importance sampling are shown in table(blah3).


\begin{table}[H]
  \centering
  \caption{Numerically calculated laplacian.}
  \begin{tabular}{ | c | c | c | c | c | c |}
    \hline
    N &D &Analytical &Numerical &Variance &Time [s] \\*
    \hline
    1& 1& 0.5 & 0.5 &3.6e-16 &0.109 \\*
    \hline
    1& 2& 1& 1&  2.6e-15& 0.120 \\*
    \hline
    1& 3& 1.5& 1.5& 1.6e-14& 0.135 \\*
    \hline
    10& 1& 5& 5& 1.3e-13& 0.498 \\*
    \hline
    10& 2& 10 &10 &1.7e-13 &0.706 \\*
    \hline
    10 &3 &15 &15 &4.3e-13 &1.07 \\*
    \hline
    100& 1& 50& 50& 1.1e-10& 8.55  \\*
    \hline
    100& 2& 100& 100& 2.4e-09& 20.9 \\*
    \hline
    100& 3& 150 &150 &6.8e-09& 39.5  \\*
    \hline
    500& 1& 250& 250& 3.4e-08& 118  \\*
    \hline
    500& 2& 500& 500& 3.9e-07& 400  \\*
    \hline
    500& 3& 750 &750 &3.0e-06 &834   \\*
    \hline
  \end{tabular}
  \label{tab:Tabell1}
\end{table}


\begin{table}[H]
  \centering
  \caption{Analytically calculated laplacian.}
  \begin{tabular}{ | c | c | c | c | c | c |}
    \hline
    N &D &Analytical &Numerical &Variance &Time [s] \\*
    \hline
    1& 1& 0.5 & 0.5 &0 &0.09 \\*
    \hline
    1& 2& 1& 1&  0& 0.10\\*
    \hline
    1& 3& 1.5& 1.5& 0& 0.13 \\*
    \hline
    10& 1& 5& 5& 0& 0.33 \\*
    \hline
    10& 2& 10 &10 &0 &0.37 \\*
    \hline
    10 &3 &15 &15 &0 &0.38 \\*
    \hline
    100& 1& 50& 50& 0& 2.92  \\*
    \hline
    100& 2& 100& 100& 0& 2.90 \\*
    \hline
    100& 3& 150 &150 &0& 2.95  \\*
    \hline
    500& 1& 250& 250& 0& 14.2  \\*
    \hline
    500& 2& 500& 500& 0&14.5   \\*
    \hline
    500& 3& 750 &750 &0 &15.2   \\*
    \hline
  \end{tabular}
  \label{tab:Tabell1}
\end{table}










\begin{table}[H]
  \centering
  \caption{some caption}
  \begin{tabular}{ | c | r | r | r | r | r |}
    \hline
    N &D &Analytical &Numerical &Variance &Time \\*
    \hline
    1& 1&   \\*
    \hline
    1& 2&   \\*
    \hline
    1& 3&  \\*
    \hline
    10& 1&  \\*
    \hline
    10& 2&\\*
    \hline
    10 &3&  \\*
    \hline
    100& 1&  \\*
    \hline
    100& 2&  \\*
    \hline
    100& 3&  \\*
    \hline
    500& 1&  \\*
    \hline
    500& 2&  \\*
    \hline
    500& 3&   \\*
    \hline
  \end{tabular}
  
  \label{tab:Tabell1}
\end{table}

\section{Conclusions}


\section{Appendix}


 %\end{multicols*}
%%%%%%%%%%%%%%%%%%%%%%%%%%%%%%%%%%%
%%%%%%%%%%%%%%%%%%%%%%%%%%%%%%%%%%%
\end{document}

\begin{comment}

% deloppgave
\begin{enumerate}
\item[\bf a)]
\item[\bf b)]
\item[\bf c)]
\item[\bf d)]
\end{enumerate}

%%%%%%%%
% Tabell
\begin{table}[H]
  \centering
  \begin{tabular}{ | c | r | r | r | r | r |}
    \hline
    & & & & & \\*
    \hline
    & & & & & \\*
    \hline
  \end{tabular}
  \caption{some caption}
  \label{tab:Tabell1}
\end{table}

%%%%%%%%
% Enkel figur
\begin{figure}[H]
\begin{center}
  \includegraphics[width = 120mm]{/users/filiphl/Desktop/Studie/Emne/ObligX/filnavn.png}
  \caption{some caption}\label{fig:fig1}
  \end{center}
\end{figure}

%%%%%%%%
% 2 figurer sbs
\begin{minipage}[t]{0.48\linewidth}
  \includegraphics[width=\textwidth]{fil}
  \caption{}
  \label{fig:minipage1}
\end{minipage}
\quad
\begin{minipage}[t]{0.48\linewidth}
\includegraphics[width=\textwidth]{fil}
  \caption{}
  \label{fig:minipage1}
\end{minipage}
\end{figure}

%%%%%%%%
% X antall kollonner
\begin{multicols*}{X}
\begin{spacing}{0.7} % verticale mellomrom
%kan f.eks benytte align?
\end{spacing}
\end{multicols*}


%%%%%%%%
%Matrise
\begin{equation*}
    {\bf A} = \left(\begin{array}{cccccc}
                           z &z &z &z &z &z \\
                           z &z &z &z &z &z \\
                           z &z &z &z &z &z \\
                           z &z &z &z &z &z \\
                           z &z &z &z &z &z \\
                           z &z &z &z &z &z \\
                      \end{array} \right)
\end{equation*}
%%%%%%%%

