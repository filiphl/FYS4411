\documentclass[english, a4paper]{article}

\usepackage[T1]{fontenc}    % Riktig fontencoding
\usepackage[utf8]{inputenc} % Riktig tegnsett
\usepackage{babel}          % Ordelingsregler, osv
\usepackage{graphicx}       % Inkludere bilder
\usepackage{booktabs}       % Ordentlige tabeller
\usepackage{url}            % Skrive url-er
\usepackage{textcomp}       % Den greske bokstaven micro i text-mode
\usepackage{units}          % Skrive enheter riktig
\usepackage{float}          % Figurer dukker opp der du ber om
\usepackage{lipsum}         % Blindtekst
\usepackage{subcaption} 
\usepackage{color}
\usepackage{amsmath}  
\usepackage{braket} 
\usepackage{multicol}
\usepackage{listings}    %Add source code
\usepackage{amsfonts}
\usepackage{setspace}
\usepackage[cm]{fullpage}		% Smalere marger.
\usepackage{verbatim} % kommentarfelt.
\setlength{\columnseprule}{1pt}	%(width of separationline)
\setlength{\columnsep}{1.0cm}	%(space from separation line)
\newcommand\lr[1]{\left(#1\right)} 
\newcommand\bk[1]{\langle#1\rangle} 
\newcommand\uu[1]{\underline{\underline{#1}}} % Understreker dobbelt.
\definecolor{qc}{rgb}{0,0.4,0}


% JF i margen
\makeatletter
\renewcommand{\subsubsection}{\@startsection{subsubsection}{3}{-2cm}%
{-\baselineskip}{0.5\baselineskip}{\bf\large}}
\makeatother
\newcommand{\jf}[1]{\subsubsection*{JF #1}\vspace*{-2\baselineskip}}

% Skru av seksjonsnummerering (-1)
\setcounter{secnumdepth}{3}

\begin{document}
\renewcommand{\figurename}{Figure}
% Forside
\begin{titlepage}
\begin{center}

\textsc{\Large FYS4411 - Computational quantum mechanics }\\[0.5cm]
\textsc{\Large Spring 2016}\\[1.5cm]
\rule{\linewidth}{0.5mm} \\[0.4cm]
{ \huge \bfseries  Project 1}\\[0.10cm]
\rule{\linewidth}{0.5mm} \\[1.5cm]
\textsc{\Large temporary report}\\[1.5cm]


% Av hvem?
\begin{minipage}{\textwidth}
\begin{minipage}{0.49\textwidth}
    \begin{center} \large
        Sean Bruce Sangolt Miller\\
        {\footnotesize s.b.s.miller@fys.uio.no}
    \end{center}
\end{minipage}
\quad
\begin{minipage}{0.49\textwidth}
    \begin{center} \large
        Filip Henrik Lasren\\
        {\footnotesize filiphenriklarsen@gmail.com}
    \end{center}
\end{minipage}
\end{minipage}
\vfill

% Dato nederst
\large{Date: \today}

\end{center}
\end{titlepage}
%%%%%%%%%%%%%%%%%%%%%%%%%%%%%%%%%%%

\begin{abstract}
 Some text that is abstact
\end{abstract}


%%%%%%%%%%%%%%%%%%%%%%%%%%%%%%%%%%%
\pagenumbering{gobble}% Remove page numbers (and reset to 1)
\tableofcontents
\newpage
\pagenumbering{arabic}% Arabic page numbers (and reset to 1)
%\begin{multicols*}{2}


\section{Introduction}


\section{Theory and methods}



\begin{equation}
    E_L({\bf R})=\frac{1}{\Psi_T({\bf R})}H\Psi_T({\bf R}),
    \label{eq:locale}
 \end{equation}
 
 \begin{equation}
   E_L({\bf R})=\frac{1}{\Psi_T({\bf R})}\sum_i^N \left(
	 \frac{-\hbar^2}{2m}
	 { \bigtriangledown }_{i}^2 +
	\frac{1}{2}m\omega_{ho}^2r_i^2\right)\Psi_T({\bf R}),
    \label{eq:localeInsertH}
 \end{equation}
The potential term is trivial since this is a scalar. A more challenging problem is to find an expression for $\nabla^2_i\Psi_T({\bf R})$.
With our wavefunction given as
\begin{equation}
 \Psi_T({\bf R}) = \prod_i e^{-\alpha r_i^2}	\label{wavefunction_a}
\end{equation}
we may start by taking the first derivatives 
\begin{align}
 \nabla_j \prod_i e^{-\alpha r_i^2} 
 &= -2\alpha r_j e^{-\alpha r_j^2} \prod_{i \neq j} e^{-\alpha r_i^2} \\
 &= -2\alpha r_j  \prod_i e^{-\alpha r_i^2}.
\end{align}
The second derivatives are then
\begin{align}
 \nabla_j^2 \prod_i e^{-\alpha r_i^2} 
 &= \nabla_j \lr{-2\alpha r_j  \prod_i e^{-\alpha r_i^2}}\\
 &= \lr{4\alpha^2 r_j^2 - 2\alpha}  \prod_i e^{-\alpha r_i^2}.
\end{align}
Inserting this into our local energy \eqref{eq:localeInsertH} we have
\begin{align}
 E_L({\bf R})=\frac{1}{\Psi_T({\bf R})}\sum_i^N \left(
	 \frac{-\hbar^2}{2m}
	 { \bigtriangledown }_{i}^2 +
	\frac{1}{2}m\omega_{ho}^2r^2\right)\Psi_T({\bf R}),
    \label{eq:localeInsertedDoubleDerivative}
\end{align}










\subsection{Preliminary derivations}












\section{Results}


\section{Conclusions}


\section{Appendix}


 %\end{multicols*}
%%%%%%%%%%%%%%%%%%%%%%%%%%%%%%%%%%%
%%%%%%%%%%%%%%%%%%%%%%%%%%%%%%%%%%%
\end{document}

\begin{comment}

% deloppgave
\begin{enumerate}
\item[\bf a)]
\item[\bf b)]
\item[\bf c)]
\item[\bf d)]
\end{enumerate}

%%%%%%%%
% Tabell
\begin{table}[H]
  \centering
  \begin{tabular}{ | c | r | r | r | r | r |}
    \hline
    & & & & & \\*
    \hline
    & & & & & \\*
    \hline
  \end{tabular}
  \caption{some caption}
  \label{tab:Tabell1}
\end{table}

%%%%%%%%
% Enkel figur
\begin{figure}[H]
\begin{center}
  \includegraphics[width = 120mm]{/users/filiphl/Desktop/Studie/Emne/ObligX/filnavn.png}
  \caption{some caption}\label{fig:fig1}
  \end{center}
\end{figure}

%%%%%%%%
% 2 figurer sbs
\begin{minipage}[t]{0.48\linewidth}
  \includegraphics[width=\textwidth]{fil}
  \caption{}
  \label{fig:minipage1}
\end{minipage}
\quad
\begin{minipage}[t]{0.48\linewidth}
\includegraphics[width=\textwidth]{fil}
  \caption{}
  \label{fig:minipage1}
\end{minipage}
\end{figure}

%%%%%%%%
% X antall kollonner
\begin{multicols*}{X}
\begin{spacing}{0.7} % verticale mellomrom
%kan f.eks benytte align?
\end{spacing}
\end{multicols*}


%%%%%%%%
%Matrise
\begin{equation*}
    {\bf A} = \left(\begin{array}{cccccc}
                           z &z &z &z &z &z \\
                           z &z &z &z &z &z \\
                           z &z &z &z &z &z \\
                           z &z &z &z &z &z \\
                           z &z &z &z &z &z \\
                           z &z &z &z &z &z \\
                      \end{array} \right)
\end{equation*}
%%%%%%%%

