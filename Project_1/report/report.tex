\documentclass[english, a4paper]{article}

\usepackage[T1]{fontenc}    % Riktig fontencoding
\usepackage[utf8]{inputenc} % Riktig tegnsett
\usepackage{babel}          % Ordelingsregler, osv
\usepackage{graphicx}       % Inkludere bilder
\usepackage{booktabs}       % Ordentlige tabeller
\usepackage{url}            % Skrive url-er
\usepackage{textcomp}       % Den greske bokstaven micro i text-mode
\usepackage{units}          % Skrive enheter riktig
\usepackage{float}          % Figurer dukker opp der du ber om
\usepackage{lipsum}         % Blindtekst
\usepackage{subcaption} 
\usepackage{color}
\usepackage{amsmath}  
\usepackage{braket} 
\usepackage{multicol}
\usepackage{listings}    %Add source code
\usepackage{amsfonts}
\usepackage{setspace}
\usepackage[cm]{fullpage}		% Smalere marger.
\usepackage{verbatim} % kommentarfelt.
\setlength{\columnseprule}{1pt}	%(width of separationline)
\setlength{\columnsep}{1.0cm}	%(space from separation line)
\newcommand\lr[1]{\left(#1\right)} 
\newcommand\bk[1]{\langle#1\rangle} 
\newcommand\uu[1]{\underline{\underline{#1}}} % Understreker dobbelt.
\definecolor{qc}{rgb}{0,0.4,0}


% JF i margen
\makeatletter
\renewcommand{\subsubsection}{\@startsection{subsubsection}{3}{0pt}%
{-\baselineskip}{0.5\baselineskip}{\bf\large}}
\makeatother
\newcommand{\jf}[1]{\subsubsection*{JF #1}\vspace*{-2\baselineskip}}

\newcommand{\bm}[1]{\mathbf{#1}}

% Skru av seksjonsnummerering (-1)
\setcounter{secnumdepth}{3}

\begin{document}
\renewcommand{\figurename}{Figure}
% Forside
\begin{titlepage}
\begin{center}

\textsc{\Large FYS4411 - Computational quantum mechanics }\\[0.5cm]
\textsc{\Large Spring 2016}\\[1.5cm]
\rule{\linewidth}{0.5mm} \\[0.4cm]
{ \huge \bfseries  Project 1;\\ Variational Monte Carlo Studies of Bosonic systems}\\[0.10cm]
\rule{\linewidth}{0.5mm} \\[1.5cm]
\textsc{\Large temporary report}\\[1.5cm]


% Av hvem?
\begin{minipage}{\textwidth}
\begin{minipage}{0.49\textwidth}
    \begin{center} \large
        Sean Bruce Sangolt Miller\\
        {\footnotesize s.b.s.miller@fys.uio.no}
    \end{center}
\end{minipage}
\quad
\begin{minipage}{0.49\textwidth}
    \begin{center} \large
        Filip Henrik Lasren\\
        {\footnotesize filiphenriklarsen@gmail.com}
    \end{center}
\end{minipage}
\end{minipage}
\vfill

% Dato nederst
\large{Date: \today}

\end{center}
\end{titlepage}
%%%%%%%%%%%%%%%%%%%%%%%%%%%%%%%%%%%

\begin{abstract}
 Some text that is abstact
\end{abstract}


%%%%%%%%%%%%%%%%%%%%%%%%%%%%%%%%%%%
\pagenumbering{gobble}% Remove page numbers (and reset to 1)
\tableofcontents
\newpage
\pagenumbering{arabic}% Arabic page numbers (and reset to 1)
%\begin{multicols*}{2}


\section{Introduction}


\section{Theory and methods}
\subsection{Preliminary derivations}

\subsubsection{Simplified problem}
The local energy is defined as:

\begin{equation}
    E_L({\bf R})=\frac{1}{\Psi_T({\bf R})}H\Psi_T({\bf R}),
    \label{eq:locale}
\end{equation}

As a first approximation, it is assumed there is no interaction term in the Hamiltonian, which means the hard sphere bosons have no physical size (the hard-core diameter is zero). It is also assumed that no magnetic field is applied to the bosonic gas, leaving a perfectly spherically symmetrical harmonic trap. Inserting this new Hamiltonian into the local energy gives:

\begin{equation}
  E_L({\bf R})=\frac{1}{\Psi_T({\bf R})}\sum_i^N \left(
  \frac{-\hbar^2}{2m}
  \nabla_{i}^2 +
  \frac{1}{2}m\omega_{ho}^2r_i^2\right)\Psi_T({\bf R})
  \label{eq:localeInsertH}
\end{equation}

The potential term is trivial since this is a scalar, i.e. the denominator will cancel the wavefunction. A more challenging problem is to find an expression for $\nabla^2_i\Psi_T({\bf R})$. The trial wavefunction shown in equation (...), with the aforementioned approximations, is:

\begin{equation}
 \Psi_T({\bf R}) = \prod_i e^{-\alpha r_i^2}	\label{wavefunction_a}
\end{equation}

where $\alpha$ is the variational parameter for VCM. The first derivative is:

\begin{align}
 \nabla_j\prod_i e^{-\alpha r_i^2} 
 &= -2\alpha \bm{r}_j e^{-\alpha r_j^2} \prod_{i \neq j} e^{-\alpha r_i^2}\\
 &= -2\alpha \bm{r}_j  \prod_i e^{-\alpha r_i^2}.
 \label{E_L_first_derivative}
\end{align}

The second derivative then follows:

\begin{align}
 \nabla_j^2 \prod_i e^{-\alpha r_i^2} 
 &= \nabla_j \lr{-2\alpha \bm{r}_j  \prod_i e^{-\alpha r_i^2}}\\
 &= \lr{4\alpha^2 r_j^2 - 2d\alpha}  \prod_i e^{-\alpha r_i^2}.
\end{align}

where $d$ is the number of dimensions. Inserting this into back into the local energy (equation \eqref{eq:localeInsertH}), the final expression can be derived:

\begin{align*}
 E_L({\bf R}) &= \frac{1}{\Psi_T({\bf R})}\sum_i^N \lr{
	 \frac{-\hbar^2}{2m}
	 \nabla_{i}^2 +
	\frac{1}{2}m\omega_{ho}^2r^2}\Psi_T({\bf R})\\
    &= \sum_{i=1}^N\left[\frac{-\hbar^2}{2m}\lr{4\alpha^2 r_i^2 - 2d\alpha} + \frac{1}{2}m\omega_{ho}^2r_i^2\right]
\end{align*}



The drift force (quantum force), still with the approximations above, is defined by:

\begin{equation}
	F = \frac{2\nabla\Psi_T}{\Psi_T}
\end{equation}

The gradient here is defined as

\begin{equation*}
	\nabla \equiv \lr{\nabla_1,\nabla_2,\ldots,\nabla_N}
\end{equation*}

i.e. a vector of dimension $Nd$. The gradient with respect to a single particle's position is already given in equation \ref{E_L_first_derivative}, so it's not too hard to see the following is the necessary factor in the drift force:

\begin{align*}
	F &= \frac{-4\alpha}{\Psi_T}\lr{\bm{r}_1,\bm{r}_2,\ldots,\bm{r}_N}\Psi_T\\
	&= -4\alpha\lr{\bm{r}_1,\bm{r}_2,\ldots,\bm{r}_N}
\end{align*}


\subsubsection{Full problem}

The full problem\footnote{The "full problem" means not making any assumptions on the particle interactions or the potential.} is a bit more tedious to derive. The first step is to rewrite the trial wavefunction to the following form:

\begin{equation}
	\Psi_T(\bm{R}) = \prod_{i} \phi(\bm{r}_i)e^{\sum_{i'<j'}u(r_{i'j'})}
\end{equation}

where, in order for this to fit with the previous wavefunction, $u(r_{ij}) \equiv \ln(f(r_{ij}))$ and $\phi(\bm{r}_i) \equiv g(\alpha,\beta,\bm{r}_i)$. The gradient with respect to the $k$-th coordinate set is:

\begin{align}
	\nabla_k\Psi_T &= \nabla_k\prod_i \phi(\bm{r}_i)e^{\sum_{i'<j'}u(r_{i'j'})}\\
	&= \nabla_k\phi_k\left[\prod_{i\neq k} \phi(\bm{r}_i)e^{\sum_{i'<j'}u(r_{i'j'})}\right] + \left[\prod_i \phi(\bm{r}_i)e^{\sum_{i'<j'}u(r_{i'j'})}\nabla_k\left(\sum_{i''<j''}u_{i''j''} \right)\right]\\
	&= \nabla_k\phi_k\left[\prod_{i\neq k} \phi(\bm{r}_i)e^{\sum_{i'<j'}u(r_{i'j'})}\right] + \left[\prod_i \phi(\bm{r}_i)e^{\sum_{i'<j'}u(r_{i'j'})}\left(\sum_{i''<j''}\nabla_ku_{i''j''} \right)\right]
\end{align}

The function $u_{ij}$ is symmetric under permutation of $i$ and $j$, as one can see from the definitions of itself and $f(r_{ij})$. This means that in the last sum above, one can always s

\section{Results}

\begin{table}[H]
  \centering
  \begin{tabular}{ | c | r | r | r | r | r |}
    \hline
    N &D &Analytical &Numerical &Variance &Time \\*
    \hline
    1& 1& 0  \\*
    \hline
    1& 2& 0  \\*
    \hline
    1& 3& 0.88 \\*
    \hline
    10& 1&  \\*
    \hline
    10& 2& 10 &10 &7.4e-13 &1.35 \\*
    \hline
    10 &3 &15 &15 &3.7e-13 &1.41 \\*
    \hline
    100& 1&  \\*
    \hline
    100& 2&  \\*
    \hline
    100& 3& 150 &150 &8.5e-9 &119.38  \\*
    \hline
    500& 1&  \\*
    \hline
    500& 2&  \\*
    \hline
    500& 3& 250 &250 &4.1e-8 &316.19   \\*
    \hline
  \end{tabular}
  \caption{some caption}
  \label{tab:Tabell1}
\end{table}

\section{Conclusions}


\section{Appendix}


 %\end{multicols*}
%%%%%%%%%%%%%%%%%%%%%%%%%%%%%%%%%%%
%%%%%%%%%%%%%%%%%%%%%%%%%%%%%%%%%%%
\end{document}

\begin{comment}

% deloppgave
\begin{enumerate}
\item[\bf a)]
\item[\bf b)]
\item[\bf c)]
\item[\bf d)]
\end{enumerate}

%%%%%%%%
% Tabell
\begin{table}[H]
  \centering
  \begin{tabular}{ | c | r | r | r | r | r |}
    \hline
    & & & & & \\*
    \hline
    & & & & & \\*
    \hline
  \end{tabular}
  \caption{some caption}
  \label{tab:Tabell1}
\end{table}

%%%%%%%%
% Enkel figur
\begin{figure}[H]
\begin{center}
  \includegraphics[width = 120mm]{/users/filiphl/Desktop/Studie/Emne/ObligX/filnavn.png}
  \caption{some caption}\label{fig:fig1}
  \end{center}
\end{figure}

%%%%%%%%
% 2 figurer sbs
\begin{minipage}[t]{0.48\linewidth}
  \includegraphics[width=\textwidth]{fil}
  \caption{}
  \label{fig:minipage1}
\end{minipage}
\quad
\begin{minipage}[t]{0.48\linewidth}
\includegraphics[width=\textwidth]{fil}
  \caption{}
  \label{fig:minipage1}
\end{minipage}
\end{figure}

%%%%%%%%
% X antall kollonner
\begin{multicols*}{X}
\begin{spacing}{0.7} % verticale mellomrom
%kan f.eks benytte align?
\end{spacing}
\end{multicols*}


%%%%%%%%
%Matrise
\begin{equation*}
    {\bf A} = \left(\begin{array}{cccccc}
                           z &z &z &z &z &z \\
                           z &z &z &z &z &z \\
                           z &z &z &z &z &z \\
                           z &z &z &z &z &z \\
                           z &z &z &z &z &z \\
                           z &z &z &z &z &z \\
                      \end{array} \right)
\end{equation*}
%%%%%%%%

