%\documentclass{article}
%\usepackage[margin=2cm]{geometry}
%\begin{document}

In column number 4 of table \ref{tab:bench} the ratio shows a linear dependece between 
number of processes and achieved speed up, which can be seen more clearly 
in the graph below. Column 6 and 9 shows the speed up due to vectorization flags.
There is not a large difference in speed up so the justification to use O3 would 
be for heavier runs than what were tested here. And since the $O3$ flag appends
more compile options it could be more invasive and create errors in the compiled
code. Hence we are better of using $O2$ in the small scale calculations.

From the graph in figure \ref{fig:bench} to the right there is a clear linear 
dependence on the number of processes. Due to the simple nature of this kind of parallizations there is very
little time consumption regarding overhead and therefore linearity is to be expected.
Also there is a danger of taking too many processes when not using enough metropolis
cycles considering the number of cycles are divided to each process and therefore 
can reach too few cycles to have a good result.
The difference in speed up due to vectorization $O2$ and $O3$ is about 0.05 independent 
of number of processes, but they both go down as the number increases. The exact nature
of this is not known, but it a possible explanation is that the amount of
time spent going through loops on each process effects vectorization speed up.
%\end{document}
