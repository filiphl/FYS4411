\documentclass[english, a4paper]{article}

\usepackage[T1]{fontenc}    % Riktig fontencoding
\usepackage[utf8]{inputenc} % Riktig tegnsett
\usepackage{babel}          % Ordelingsregler, osv
\usepackage{graphicx}       % Inkludere bilder
\usepackage{booktabs}       % Ordentlige tabeller
\usepackage{url}            % Skrive url-er
\usepackage{textcomp}       % Den greske bokstaven micro i text-mode
\usepackage{units}          % Skrive enheter riktig
\usepackage{float}          % Figurer dukker opp der du ber om
\usepackage{lipsum}         % Blindtekst
\usepackage{subcaption} 
\usepackage{color}
\usepackage{amsmath}  
\usepackage{hyperref}
\usepackage{pagecolor}
%\usepackage{minted}
\usepackage{braket} 
\usepackage{multicol}
\usepackage{listings}    %Add source code
\usepackage{amsfonts}
\usepackage{setspace}
\usepackage[cm]{fullpage}		% Smalere marger.
\usepackage{verbatim} % kommentarfelt.
\usepackage{tabularx}
\usepackage{booktabs}
\setlength{\columnseprule}{1pt}	%(width of separationline)
\setlength{\columnsep}{1.0cm}	%(space from separation line)
\newcommand\lr[1]{\left(#1\right)} 
\newcommand\bk[1]{\langle#1\rangle} 
\newcommand\uu[1]{\underline{\underline{#1}}} % Understreker dobbelt.
\definecolor{qc}{rgb}{0,0.4,0}
\definecolor{LightBlue}{rgb}{0.8, 0.8, 0.9}
\hypersetup{
	colorlinks,
	linkcolor={red!30!black},
	citecolor={blue!50!black},
	urlcolor={blue!80!black}
}
% JF i margen
\makeatletter
\renewcommand{\subsubsection}{\@startsection{subsubsection}{3}{0pt}%
{-\baselineskip}{0.5\baselineskip}{\bf\large}}
\makeatother
\newcommand{\jf}[1]{\subsubsection*{JF #1}\vspace*{-2\baselineskip}}

\newcommand{\bm}[1]{\mathbf{#1}}

% Skru av seksjonsnummerering (-1)
\setcounter{secnumdepth}{3}

\begin{document}
%\pagecolor{black!50!}
\renewcommand{\figurename}{Figure}
% Forside
\begin{titlepage}
\begin{center}

\textsc{\Large FYS4411 - Computational quantum mechanics }\\[0.5cm]
\textsc{\Large Spring 2016}\\[1.5cm]
\rule{\linewidth}{0.5mm} \\[0.4cm]
{ \huge \bfseries  Project 2;\\ Variational Monte Carlo studies of electronic systems}\\[0.10cm]
\rule{\linewidth}{0.5mm} \\[1.5cm]

{\Large Github repository:} \\*[0.4cm]
\url{https://github.com/filiphl/FYS4411.git}

\vspace{13.5cm}

% Av hvem?
\begin{minipage}{\textwidth}
\begin{minipage}{0.49\textwidth}
    \begin{center} \large
        Sean Bruce Sangolt Miller\\
        {\footnotesize s.b.s.miller@fys.uio.no}
    \end{center}
\end{minipage}
\quad
\begin{minipage}{0.49\textwidth}
    \begin{center} \large
        Filip Henrik Larsen\\
        {\footnotesize filiphenriklarsen@gmail.com}
    \end{center}
\end{minipage}
\end{minipage}
\vfill

% Dato nederst
\large{Date: \today}

\end{center}
\end{titlepage}
%%%%%%%%%%%%%%%%%%%%%%%%%%%%%%%%%%%

\begin{abstract}
	
\end{abstract}


%%%%%%%%%%%%%%%%%%%%%%%%%%%%%%%%%%%
\pagenumbering{gobble}% Remove page numbers (and reset to 1)
\tableofcontents
\newpage
\pagenumbering{arabic}% Arabic page numbers (and reset to 1)
%\begin{multicols*}{2}


\section{Introduction}


\section{Theory and Methods}
\subsection{Preliminary derivations}
While performing VMC it is of course favourable to use analytical expressions, should they not demand a significant increase in CPU time. We will therefore need to calculate the local energy $E_L = \frac{1}{\Psi_T}H\Psi_T$ and the quantum force $F = \frac{2}{\Psi_T}\nabla\Psi_T$. The Hamiltonian $H$ used will be:

\begin{equation}
	H = H_0 + H_I = \sum_{i=1}^{N}\left(-\frac{1}{2}\nabla_i^2 + \frac{1}{2}\omega^2r_i^2\right) + \sum_{i<j}\frac{1}{r_{ij}}
\end{equation}

The Laplacian will be the most demanding quantity to calculate.

\subsubsection{Singlet electron state}
For an electron in a harmonic oscillator potential, the energy is given by $\epsilon_n = \omega(n + 1)$, where we have used natural units and $n = n_x + n_y + \ldots$. For two non-interacting electrons, the energy is $\epsilon_{n_1,n_2} = \omega(n_{x,1} + n_{y,1} + \ldots + n_{x,2} + n_{y,2} + \ldots + 2)$. Obviously the energy is lowest for $n_1 = n_2 = 0$, giving $\epsilon_{0,0} = 2\omega$.\\
Since $n_1=n_2 = 0$ means the two electron are in the same spatial wavefunction, they must have different spins. 
Since electrons are spin-$\frac{1}{2}$ particles, they combine to give total spin zero, i.e. they form the singlet state.\\

For the singlet electron state we will use the trial wavefunction:

\begin{equation}
	\Psi_T(\bm{r}_1,\bm{r}_2) = Ce^{-\frac{\alpha\omega}{2}(r_1^2+r_2^2)}e^{\frac{ar_{12}}{1+\beta r_{12}}}
\end{equation}

The Laplacian of which (for particle $i$) is:

\begin{equation}
	\nabla_i^2 \Psi_T = \nabla_i(\nabla_i\Psi_T)
\end{equation}

We will use the following change of coordinates when it simplifies calculations.

\begin{align}
	\begin{split}
	\frac{\partial}{\partial r_{i,j}} &= \frac{\partial r_{12}}{\partial r_{i,j}}\frac{\partial}{\partial r_{12}}\\
	&= \frac{(-1)^i}{r_{12}}(x_1-x_2, y_1-y_2)\frac{\partial}{\partial r_{12}}\\
	&= \frac{(-1)^i}{r_{12}}\bm{r}_{12}\frac{\partial}{\partial r_{12}}
	\end{split}
\end{align}

Where $r_{i,j}$ is element $j$ of $r_i$.
The gradient, which is also needed for the quantum force, is then:

\begin{align}
	\begin{split}
	\nabla_i\Psi_T &= -\alpha\omega\bm{r}_i\Psi_T + \frac{(-1)^i}{r_{12}}\bm{r}_{12}\left[\frac{\partial}{\partial r_{12}}\left(\frac{ar_{12}}{1+\beta r_{12}}\right)\right]\Psi_T\\
	&= \left[-\alpha\omega\bm{r}_i + \frac{(-1)^i}{r_{12}}\bm{r}_{12}\frac{a}{(1+\beta r_{12})^2}\right]\Psi_T
	\end{split}
	\label{eq:grad_singlet}
\end{align} 

which means the Laplacian is:

\begin{equation}
	\begin{split}
	\nabla_i^2\Psi_T &= \left[\nabla_i[\ldots]\right]\Psi_T + [\ldots]\nabla_i\Psi_T\\
	&= \left[\nabla_i[\ldots]\right]\Psi_T + [\ldots]^2\Psi_T
	\end{split}
\end{equation}

where $[\ldots]$ is the last parenthesis in equation \ref{eq:grad_singlet}. The parenthesis in the first term above is:

\begin{align}
	\begin{split}
	\nabla_i\left[-\alpha\omega\bm{r}_i + \frac{(-1)^i}{r_{12}}\bm{r}_{12}\frac{a}{(1+\beta r_{12})^2}\right] &= -2\alpha\omega + \frac{(-1)^i}{r_{12}}\left( \frac{(-1)^i2ar_{12}}{(1+\beta)^2} - \frac{(-1)^i2a\beta r_{12}}{(1+\beta r_{12})^3} - \frac{(-1)^ia}{r_{12}(1+\beta r_{12})^2}\right)\\
	&= -2\alpha\omega - \frac{a}{(1+\beta r_{12})^2}\left( \frac{1}{r_{12}} - \frac{2}{r_{12}} + \frac{2\beta}{1+\beta r_{12}}\right)\\
	&= -2\alpha\omega + \frac{a}{r_{12}(1+\beta r_{12})^2}- \frac{2a\beta}{(1+\beta r_{12})^3}
	\end{split}
\end{align}

Which gives:

\begin{equation}
	\nabla_i^2\Psi_T = \left[-2\alpha\omega + \frac{a}{r_{12}(1+\beta r_{12})^2} - \frac{2a\beta}{(1+\beta r_{12})^3} + \alpha^2\omega^2r_i^2 + \frac{a^2}{(1+\beta r_{12})^4} - \frac{2\alpha\omega a(-1)^i}{r_{12}(1+\beta r_{12})^2}\bm{r}_i\cdot\bm{r}_{12}\right] \Psi_T
\end{equation}

We therefore have:

\begin{equation}
	\sum_{i=1}^2\frac{1}{\Psi_T}\nabla_i^2\Psi_T = -4\alpha\omega + \frac{2a}{r_{12}(1+\beta r_{12})^2} - \frac{4a\beta}{(1+\beta r_{12})^3} + \alpha^2\omega^2(r_1^2 + r_2^2) + \frac{2a^2}{(1+\beta r_{12})^4} - \frac{2\alpha\omega a}{(1+\beta r_{12})^2}r_{12}
\end{equation}

\subsubsection{Filled, 2-dimensional shell states}
If we again consider the non-interacting, harmonic oscillator confined electron system, we can increase the number of electrons beyond 2. If, in some wondrous universe, we had fermions with three different spins states (like a spin-1 boson), then for three electrons we could again set $n_1=n_2=n_3 = 0$ and have an anti-symmetric spin state. However, in our world, we must go up in energy for more than two electrons.\\
The $n=0$ state was the ground state. The next state has degeneracy 2; $(n_1,n_2) = (1,0), (0,1)$. After that we have degeneracy 3;$(n_1,n_2) = (2,0), (1,1), (0,2)$. Each of these states are doubly degenerate due to spin.\\

The reason in explaining this is because, assuming we have a so-called "full shell" problem\footnote{For every new tier in energy, we fill it up with electrons. So all considered tiers, or "shells", are full.}, we can do some manipulations that greatly reduce the number of calculations necessary to perform VMC. Firstly, we need to rewrite the total wavefunction.\\
As is already known, the true wavefunction is approximated by an analytical solution to some simpler problem, and a Jastrow factor. The analytical part can be written as a Slater determinant:

\begin{equation}
	\Psi_{D} = \frac{1}{\sqrt{N}}
		\begin{vmatrix}
		\phi_1(\bm{r}_1) & \phi_2(\bm{r}_1) & \ldots & \phi_{N-1}(\bm{r}_1) & \phi_N(\bm{r}_1)\\
		\phi_1(\bm{r}_2) & \phi_2(\bm{r}_2) & \ldots & \phi_{N-1}(\bm{r}_2) & \phi_N(\bm{r}_2)\\
		& & \vdots & &\\
		\phi_1(\bm{r}_N) & \phi_2(\bm{r}_N) & \ldots & \phi_{N-1}(\bm{r}_N) & \phi_N(\bm{r}_N)\\
		\end{vmatrix}
\end{equation}

and therefore our trial wavefunction will be:

\begin{equation}
	\Psi_T = \Psi_{D}\Psi_{C} \:\:,\:\: \Psi_{C} = \prod_{i<j}^N e^{\frac{ar_{ij}}{1+\beta"r_{ij}}}
\end{equation}

where $\alpha$, $\beta$ are the variational parameters, $a$ is connected to particle spins, and $N$ is the total number of particles.\\
Obviously, $\Psi_{D}$ is a time consuming object to calculate at every Metropolis step. We will therefore do some neat tricks that reduce the number of calculations.\\
The first is to rewrite $\Psi_D$ by using that the Hamiltonian is spin-independent. Firstly, we move all spin-up particles to the "first" $\frac{N}{2}$-positions, and the spin-down particles the remainder of positions. \\

\subsubsection{The Metropolis ratio test}
At each Metropolis step, we need the ratio of probabilities. We first define $R\equiv\frac{\Psi_T^n}{\Psi_T^o}$, where "$n$" means the new wavefunction and "$o$" means the old (or the current, but "c" could be confused with "correlation"). Written out, this is:

\begin{equation}
	R = \frac{|D_+^n|}{|D_+^o|}\frac{|D_-^n|}{|D_-^o|}\frac{\Psi_C^n}{\Psi_C^o}
\end{equation}

If we only move one position at a time, then only one row in either $D_+$ or $D_-$ will change. This means if we move a spin-up position, then $|D_-^n| = |D_-^o|$, so we need only consider one of the determinant fractions for each $R$.\\
Through some simple steps, one can show the determinant fraction ($R_{D}$) reduces to:

\begin{equation}
	R_D = \sum_{j=1}^{N/2} D_{ij}(\bm{r}^n)D_{ji}^{-1}(\bm{r}^o)
\end{equation}

where $D_{ij}^{-1}$ is element $ij$ of the inverse of $D$, $D$ is either the spin-up or spin-down determinant, and $\bm{r}_i$ is the moved position. Since $D_{ij}(\bm{r}) = \phi_j(\bm{r}_i)$, the only difficulty remaining is to find the elements of the inverse matrix.\\

The elements of an inverse matrix are given by the Sherman-Morrison formula, which, when applied to the current case, gives:

\begin{equation}
D_{kj}^{-1}(\bm{r}^n) =
\begin{cases}
D_{kj}^{-1}(\bm{r}^o) - \frac{D_{ki}^{-1}(\bm{r^o})}{R}\sum_{l=1}^{N/2}D_{il}(\bm{r}^n)D_{lj}^{-1}(\bm{r}^o) \:&\text{if}\: j\neq i\\ \vspace{1pt}\\
\frac{D_{ki}^{-1}(\bm{r^o})}{R} \:&\text{if}\: j = i \\
\end{cases}
\end{equation}

The ratio test for the correlation function is 

\section{Results}


\section{Comments}


\end{document}