\documentclass[english, a4paper]{article}

\usepackage[T1]{fontenc}    % Riktig fontencoding
\usepackage[utf8]{inputenc} % Riktig tegnsett
\usepackage{babel}          % Ordelingsregler, osv
\usepackage{graphicx}       % Inkludere bilder
\usepackage{booktabs}       % Ordentlige tabeller
\usepackage{url}            % Skrive url-er
\usepackage{textcomp}       % Den greske bokstaven micro i text-mode
\usepackage{units}          % Skrive enheter riktig
\usepackage{float}          % Figurer dukker opp der du ber om
\usepackage{lipsum}         % Blindtekst
\usepackage{subcaption} 
\usepackage{color}
\usepackage{amsmath}  
\usepackage{hyperref}
\usepackage{pagecolor}
%\usepackage{minted}
\usepackage{braket} 
\usepackage{multicol}
\usepackage{listings}    %Add source code
\usepackage{amsfonts}
\usepackage{setspace}
\usepackage[cm]{fullpage}		% Smalere marger.
\usepackage{verbatim} % kommentarfelt.
\usepackage{tabularx}
\usepackage{booktabs}
\setlength{\columnseprule}{1pt}	%(width of separationline)
\setlength{\columnsep}{1.0cm}	%(space from separation line)
\newcommand\lr[1]{\left(#1\right)} 
\newcommand\bk[1]{\langle#1\rangle} 
\newcommand\uu[1]{\underline{\underline{#1}}} % Understreker dobbelt.
\definecolor{qc}{rgb}{0,0.4,0}
\definecolor{LightBlue}{rgb}{0.8, 0.8, 0.9}
\hypersetup{
	colorlinks,
	linkcolor={red!30!black},
	citecolor={blue!50!black},
	urlcolor={blue!80!black}
}
% JF i margen
\makeatletter
\renewcommand{\subsubsection}{\@startsection{subsubsection}{3}{0pt}%
{-\baselineskip}{0.5\baselineskip}{\bf\large}}
\makeatother
\newcommand{\jf}[1]{\subsubsection*{JF #1}\vspace*{-2\baselineskip}}

\newcommand{\bm}[1]{\mathbf{#1}}

% Skru av seksjonsnummerering (-1)
\setcounter{secnumdepth}{3}

\begin{document}
%\pagecolor{black!50!}
\renewcommand{\figurename}{Figure}
% Forside
\begin{titlepage}
\begin{center}

\textsc{\Large FYS4411 - Computational quantum mechanics }\\[0.5cm]
\textsc{\Large Spring 2016}\\[1.5cm]
\rule{\linewidth}{0.5mm} \\[0.4cm]
{ \huge \bfseries  Project 2;\\ Variational Monte Carlo studies of electronic systems}\\[0.10cm]
\rule{\linewidth}{0.5mm} \\[1.5cm]

{\Large Github repository:} \\*[0.4cm]
\url{https://github.com/filiphl/FYS4411.git}

\vspace{13.5cm}

% Av hvem?
\begin{minipage}{\textwidth}
\begin{minipage}{0.49\textwidth}
    \begin{center} \large
        Sean Bruce Sangolt Miller\\
        {\footnotesize s.b.s.miller@fys.uio.no}
    \end{center}
\end{minipage}
\quad
\begin{minipage}{0.49\textwidth}
    \begin{center} \large
        Filip Henrik Larsen\\
        {\footnotesize filiphenriklarsen@gmail.com}
    \end{center}
\end{minipage}
\end{minipage}
\vfill

% Dato nederst
\large{Date: \today}

\end{center}
\end{titlepage}
%%%%%%%%%%%%%%%%%%%%%%%%%%%%%%%%%%%

\begin{abstract}
	
\end{abstract}


%%%%%%%%%%%%%%%%%%%%%%%%%%%%%%%%%%%
\pagenumbering{gobble}% Remove page numbers (and reset to 1)
\tableofcontents
\newpage
\pagenumbering{arabic}% Arabic page numbers (and reset to 1)
%\begin{multicols*}{2}


\section{Introduction}


\section{Theory and Methods}
\subsection{Preliminary derivations}
While performing VMC it is of course favourable to use analytical expressions, should they not demand a significant increase in CPU time. We will therefore need to calculate the local energy $E_L = \frac{1}{\Psi_T}H\Psi_T$ and the quantum force $F = \frac{2}{\Psi_T}\nabla\Psi_T$. The Hamiltonian $H$ used will be:

\begin{equation}
	H = H_0 + H_I = \sum_{i=1}^{N}\left(-\frac{1}{2}\nabla_i^2 + \frac{1}{2}\omega^2r_i^2\right) + \sum_{i<j}\frac{1}{r_{ij}}
\end{equation}

The Laplacian will be the most demanding quantity to calculate.

\subsubsection{Singlet electron state}
For the singlet electron state we will use the trial wavefunction:

\begin{equation}
	\Psi_T(\bm{r}_1,\bm{r}_2) = Ce^{-\frac{\alpha\omega}{2}(r_1^2+r_2^2)}e^{\frac{ar_{12}}{1+\beta r_{12}}}
\end{equation}

The Laplacian of which (for particle $i$) is:

\begin{equation}
	\nabla_i^2 \Psi_T = \nabla_i(\nabla_i\Psi_T)
\end{equation}

We will use the following change of coordinates when it greatly simplifies calculations.

\begin{align}
	\begin{split}
	\frac{\partial}{\partial r_{i,j}} &= \frac{\partial r_{12}}{\partial r_{i,j}}\frac{\partial}{\partial r_{12}}\\
	&= \frac{(-1)^i}{r_{12}}(x_1-x_2, y_1-y_2)\frac{\partial}{\partial r_{12}}\\
	&= \frac{(-1)^i}{r_{12}}\bm{r}_{12}\frac{\partial}{\partial r_{12}}
	\end{split}
\end{align}

Where $r_{i,j}$ is element $j$ of $r_i$.
The gradient, which is also needed for the quantum force, is then:

\begin{align}
	\begin{split}
	\nabla_i\Psi_T &= -\alpha\omega\bm{r}_i\Psi_T + \frac{(-1)^i}{r_{12}}\bm{r}_{12}\left[\frac{\partial}{\partial r_{12}}\left(\frac{ar_{12}}{1+\beta r_{12}}\right)\right]\Psi_T\\
	&= \left[-\alpha\omega\bm{r}_i + \frac{(-1)^i}{r_{12}}\bm{r}_{12}\frac{a}{(1+\beta r_{12})^2}\right]\Psi_T
	\end{split}
	\label{eq:grad_singlet}
\end{align} 

which means the Laplacian is:

\begin{equation}
	\begin{split}
	\nabla_i^2\Psi_T &= \left[\nabla_i[\ldots]\right]\Psi_T + [\ldots]\nabla_i\Psi_T\\
	&= \left[\nabla_i[\ldots]\right]\Psi_T + [\ldots]^2\Psi_T
	\end{split}
\end{equation}

where $[\ldots]$ is the last parenthesis in equation \ref{eq:grad_singlet}. The parenthesis in the first term above is:

\begin{align}
	\begin{split}
	\nabla_i\left[-\alpha\omega\bm{r}_i + \frac{(-1)^i}{r_{12}}\bm{r}_{12}\frac{a}{(1+\beta r_{12})^2}\right] &= -2\alpha\omega + \frac{(-1)^i}{r_{12}}\left( \frac{(-1)^i2ar_{12}}{(1+\beta)^2} - \frac{(-1)^i2a\beta r_{12}}{(1+\beta r_{12})^3} - \frac{(-1)^ia}{r_{12}(1+\beta r_{12})^2}\right)\\
	&= -2\alpha\omega - \frac{a}{(1+\beta r_{12})^2}\left( \frac{1}{r_{12}} - \frac{2}{r_{12}} + \frac{2\beta}{1+\beta r_{12}}\right)\\
	&= -2\alpha\omega + \frac{a}{r_{12}(1+\beta r_{12})^2}- \frac{2a\beta}{(1+\beta r_{12})^3}
	\end{split}
\end{align}

Which gives:

\begin{equation}
	\nabla_i^2\Psi_T = \left[-2\alpha\omega + \frac{a}{(1+\beta r_{12})^2} - \frac{2a\beta}{(1+\beta r_{12})^3} + \alpha^2\omega^2r_i^2 + \frac{a^2}{(1+\beta r_{12})^4} - \frac{2\alpha\omega a(-1)^i}{r_{12}(1+\beta r_{12})^2}\bm{r}_i\cdot\bm{r}_{12}\right] \Psi_T
\end{equation}

We therefore have:

\begin{equation}
	\sum_{i=1}^2\frac{1}{\Psi_T}\nabla_i^2\Psi_T = -4\alpha\omega + \frac{2a}{(1+\beta r_{12})^2} - \frac{4a\beta}{(1+\beta r_{12})^3} + \alpha^2\omega^2(r_1^2 + r_2^2) + \frac{2a^2}{(1+\beta r_{12})^4} - \frac{2\alpha\omega a}{(1+\beta r_{12})^2}r_{12}
\end{equation}



\subsection{}

\section{Results}


\section{Comments}


\end{document}